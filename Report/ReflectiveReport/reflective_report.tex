\documentclass{article}


\usepackage[margin=1in]{geometry}
\usepackage{fancyhdr}
\usepackage{enumerate}
\usepackage{float}
\usepackage{listings}
\usepackage{listings}

\setlength\parindent{0pt}

\renewcommand{\topfraction}{0.9}
\pagestyle{fancy}
\fancyhf{}
\rhead{Pedro Pinto da Silva, Alex Kell}
\lhead{Project 2}
\rfoot{Page \thepage}


\begin{document}

\title{Project 2 \\ CSC8622}
\author{Pedro Pinto da Silva, Alex Kell}

\maketitle

\section*{Reflective Writing Report}

\subsection*{How we found the process}

We found that the process was very methodical, with discrete steps to sort and process the data. As the project could be ran in series, the use of a make file was particularly helpful in creating an entire system to complete the task. We found that we learnt a lot from the different technologies. We also felt as if we had experienced a problem which we could expect to receive in the future, particularly in data analysis. \\

There were many different parts to the project, and therefore we were able to experiment and learn different technologies on each of the tasks.

\subsection*{What we learned about each of the technologies used in the solution}

We learnt that AWK is a very powerful tool when working with text files. Programs that would require multiple for, and while loops in bash could be written in a single line in AWK. We learnt of the limitations when dealing with specific problems, such as commons within fields for comma seperated values.  After using a specific CSV toolkit, we learnt of the importance of working with packages that are specifically designed for our problem.  We learnt about the ease of use for the shell, and the wide range of tasks that could be solved using this. 

\subsection*{What we disliked the most}



\subsection*{How we felt during and after the process}

During the process we felt as if we had a good opportunity to learn all of the different technologies, and apply them to a specific task. At times it was difficult to choose the correct programming language to use for a specific task. A wide range of solutions could be found to produce the same results with the differing languages and toolkits on offer. \\

After the process we found that we had learnt a lot about the different technologies, and would be able to confidently apply them in our future work.

\subsection*{What we found most difficult/straightforward}

We found the logic of the problem straightforward. Multiple different stages were created in the pipeline, and ran in sequential order. Each part making an important change on the previous part to create the final solution. We found that once we had set up the make file it was very simple to run all of the different parts sequentially, without the need for the tracking of files. This was especially useful to keep the program general.

\subsection*{What were our experiences using git?}

Our experiences of using git were very positive. We would both have the latest copy of the code, and could keep working on the project simultaneously or together.  It was easy to spot things that the other person had done, and we could go back in time to review code changes. We were able to instantly look at code that the other person was working on, and make updates to things that needed to be changed.

\end{document}